% coding:utf-8

\section{Layout}

\subsection{Grundlegendes}
Die Signale bei SATA haben eine Frequenz von 6 GHz. (Bei 6 GBit/s)
Daher müssen bei den Datenleitungen sowohl die Impedanz als auch die 
Leitungslänge beachtet werden. 

\subsection{Impedanz}
Die differentielle Impedanz soll $100 \Omega~\pm 20 \%$ betragen. Dazu wird das 
Modell für einen differentiellen Microstrip angewendet. 
\[ Z_d = \frac{174}{\sqrt{\varepsilon_r + 1.41}} \cdot 
\ln \left(\frac{5.98 \cdot h}{0.8 \cdot w + t}\right) \cdot 
\left(1 - 0.48 \cdot e^{-0.96 \cdot \frac{d}{h}}\right) \]
Für die Berechnung der Impedanz wird der Rechner von 
\url{http://www.mantaro.com/resources/impedance_calculator.
htm#differential_microstrip2_impedance} 
eingesetzt. 
Eine Impedanz von $100 \Omega~\pm 10 \%$ wird mit folgenden Werten erreicht: 
\begin{tabular}{@{}llll}
$w$             & Leiterbahnbreite          &   $34 mil$        \\
$d$             & Leiterbahnabstand         &   $6 mil$         \\
$t$             & Leiterbahndicke           &   $1.4 mil$       \\
$h$             & Dicke der Platine         &   $59 mil$        \\
$\varepsilon_r$ & Dielektrizitätskonstante  &   $4.2$           \\
$Z_d$           & Differentielle Impedanz   &   $104.2 \Omega$  
\end{tabular}

\subsection{Leitungslänge}
Die Längendifferenz der einzelnen Leitungen eines Signalpaares darf 5 mil 
nicht übersteigen. Im Aktuellen Design liegen folgende Leiterbahnlänge vor: 
\begin{tabular}{@{}ll}
A+  & 2141.69254 \\
A-  & 2141.35691 \\
B+  & 1695.5267 \\
B-  & 1695.86208 \\
\end{tabular}
Das ergibt folgende Differenzen: 
\begin{tabular}{@{}ll}
A   & 0.33563 mil \\
B   & 0.33538 mil \\
\end{tabular}
